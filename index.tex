\documentclass[12pt,a4paper,UTF8,hyperref,nofonts]{ctexbook}
\usepackage{fonts-external}
\usepackage{fontspec, graphicx, titlesec, natbib}
\usepackage{indentfirst, listings, xcolor, verbatim}
\usepackage{geometry}

% ctex关于章节的一些设置
\CTEXsetup[name={0x0,}]{chapter}
\CTEXsetup[number={\arabic{chapter}}]{chapter}

% 主要样式
\newpagestyle{main}{
  % pdf元信息
  \hypersetup{
    colorlinks=true,
    bookmarks=true,
    bookmarksopen=true,
    pdfpagemode=FullScreen,
    pdfstartview=fit,
    pdftitle={Linux exploit development tutorial},
    pdfauthor={Sn0rt}
  }

  % 页眉和页脚设置
  \setfoot{}{}{}
	\headrule

  % 段落首行缩进 2 字符
  \setlength{\parindent}{2em}
  
  % 段间距
	\setlength{\parskip}{0.5\baselineskip}
}
\pagestyle{main}

% 嵌入代码格式设置
\lstset{numbers = left, 
	keywordstyle = \color{blue}\bfseries,
	numberstyle = \small\color{black},
	backgroundcolor = \color{lightgray},
	basicstyle=\footnotesize,
	stepnumber = 1,
	showstringspaces=false,
	showspaces = false,
	showtabs = false,
	tabsize = 8,
	breaklines = true,
	extendedchars = false
}

% 设置章节路径

\title{Linux exploit development tutorial}
\author{Sn0rt@abc.shop.edu.cn}
\bibliographystyle{plain}
\begin{document}
\maketitle
% 前言
\newpage
\thispagestyle{empty}
{\hfil \huge \textbf{前\hspace{2em}言}}\par
\par 发现Linux下二进制学习曲线陡峭,而套路零散,于是整理编著这篇文章,来帮助感兴趣的人学习,还想结识更多对Linux二进制感兴趣的人.万事开头难,首先要感谢本文原来的的作者
sploitfun,他开始做了这件事并写出了思路,我在他的基础上进行了补充和翻译.

\par 还要要感谢phrack,乌云知识库,各种wiki上面文章的作者,这些作者和安全研究人员讲解了很多关于exploit相关技术,是大家的无私分享使很多东西变的可能,我也想把这样的分享精
神学习来.

\par 为了防止文章过于臃肿,我们讲分享讨论的话题尽量限制在Linux,x86,MIPS,ipv4范围内,我们假设读者能正常使用Linux,熟悉C语言,了解汇编语言,认识计算机专业词汇,基本体系结构
知识(栈,堆,内存之类的).如果不能因为知识储备不够,推荐0day安全\cite{0day安全},不建
议特为了某个事情把所有预先条件都修好,需要什么要用到在去学,因为不用都会忘的.我认
为技术人员的学习能力比现实技术重要.

\par 如果关于本文有什么疑问可以联系我.

\frontmatter

% 目录
\tableofcontents
\clearpage

\mainmatter

% 预备
\chapter{预备}
\par 在这个level 我将要花点时间给大家介绍基本的漏洞类型和安全机制,然后关闭全部的安全
 保护机制,学习如何在Linux下面编写最基本的exp.

 \section{安全机制}
 \par 分为两大类:编译相关(elf加固),部分编译选项控制着生成更安全的代码(损失部分性能或
 者空间),还有就说运行时的安全,都是为增加了漏洞利用的难度,不能从本质上去除软件的
 漏洞.

 \subsection{STACK CANARY}
 \par Canary 是放置在缓冲区和控制数据之间的一个words被用来检测缓冲区溢出, 如果发生缓
 冲区溢出那么第一个被修改的数据通常是canary,当其验证失败通常说明发生了栈溢出,更
 多信息参考这里
 \footnote{\url{https://en.wikipedia.org/wiki/Buffer_overflow_protection\#Canaries}}.
 \begin{lstlisting}[language=sh]
   gcc -fstack-protector
 \end{lstlisting}
 
 \subsection{NX}
 \par 在早期,指令是数据,数据也是数据,当PC指向哪里,那里的数据就会被当成指令被cpu执行,
 后来NX标志位被引入来区分指令和数据.更
 多信息参考这里\cite{Intel}
\footnote{<<Intel® 64 and IA-32 Architectures Software Developer’s Manual>> volumes 3 section 4.6}
\footnote{\url{https://en.wikipedia.org/wiki/NX_bit}}.

 \begin{lstlisting}[language=sh]
   gcc -z execstack
 \end{lstlisting}
 

 \subsection{FORTIFY}
 \par 在编译和运行时候保护glibc:
 \begin{list}{\textbullet}{%
    \setlength\topsep{0pt} \setlength\partopsep{0pt}
    \setlength\parsep{0pt} \setlength\itemsep{0pt}
  }
\item expand unbounded calls to "sprintf", "strcpy" into their "n"
  length-limited cousins when the size of a destination buffer is known
  (protects against memory overflows).
\item stop format string "\%n" attacks when the format string is in a writable \% memory segment. 
\item require checking various important function return codes and arguments (e.g.system, write, open).
\item require explicit file mask when creating new files. 
\end{list}
 \begin{lstlisting}[language=sh]
   gcc -D_FORTIFY_SOURCE=2 -O
 \end{lstlisting}
 
 \subsection{PIE}
 -fPIC:
 类似于-fpic不过克服了部分平台对偏移表尺寸的限制.
 生成可用于共享库的位置独立代码。所有的内部寻址均通过全局偏移表(GOT)完成.要确
 定一个地址,需要将代码自身的内存位置作为表中一项插入.该选项需要操作系统支持,因
 此并不是在所有系统上均有效.该选项产生可以在共享库中存放并从中加载的目标模块.
 参考链接
 \footnote{\url{https://en.wikipedia.org/wiki/Position-independent_code\#PIE}}.
 
 -fPIE:
 这选项类似于-fpic与-fPIC,但生成的位置无关代码只可以链接为可执行文件,它通常的链
 接选项是-pie.
 \begin{lstlisting}[language=sh]
   gcc -pie -fPIE
 \end{lstlisting}

 \subsection{RELRO}
\par Hardens ELF programs against loader memory area overwrites by having the loader mark any areas of the relocation table as read-only for any symbols resolved at
 load-time ("read-only relocations"). This reduces the area of possible
 GOT-overwrite-style memory corruption attacks
 \footnote{\url{http://blog.isis.poly.edu/exploitation\%20mitigation\%20techniques/exploitation\%20techniques/2011/06/02/relro-relocation-read-only/}}.

 \subsubsection{ASLR}
 \footnote{\url{https://en.wikipedia.org/wiki/Address_space_layout_randomization}}

 \section{漏洞类型}
 \subsection{栈溢出}
 \subsection{整数溢出}
 \subsection{off-by-one(stack base)}
 \subsection{格式化字符串}
 \%h(短写)
 \%n\$d(直接参数访问)
 \%n(任意内存写)
 \%s(任意内存读)

 \section{Exp开发}
 \subsection{rop}
 nop seld + shellcode + ret
 \subsection{.dtors(废弃)}
 
 \begin{lstlisting}[language=C]
   static void cleanup() __attribute__((destructor))
 \end{lstlisting}
 \subsection{覆写GOT}


% 栈
\chapter{stack}
\par 这个阶段可能要花点时间了,需要学习主流的bypass安全机制的部分手段(base
stack).
\par ret2any: 返回到任何可以执行的地方, 已知的地方
\begin{list}{\textbullet}{%
    \setlength\topsep{0pt} \setlength\partopsep{0pt}
    \setlength\parsep{0pt} \setlength\itemsep{0pt}
  }
\item stack
\item data/heap
\item text
\item library (libc)
\item code chunk (ROP)
\end{list}

\section{CANARY}
\subsection{overwriting TLS}

\section{NX}
\subsection{return-to-libc}
\footnote{\url{https://sploitfun.wordpress.com/2015/05/08/bypassing-nx-bit-using-return-to-libc/}}.\newline
\subsection{chained return-to-libc} 
\footnote{\url{https://sploitfun.wordpress.com/2015/05/08/bypassing-nx-bit-using-chained-return-to-libc/}}.\newline

\section{ASLR}
\subsection{return-to-plt}
\footnote{\url{https://sploitfun.wordpress.com/2015/05/08/bypassing-aslr-part-i/}}.\newline
\subsection{brute-force}
\footnote{\url{https://sploitfun.wordpress.com/2015/05/08/bypassing-aslr-part-ii/}}.\newline
\subsection{overwriting GOT}
\footnote{\url{https://sploitfun.wordpress.com/2015/05/08/bypassing-aslr-part-iii/}}.\newline


% 堆
\chapter{heap}

这个阶段可能要花更多的时间了,堆上面的安全一直是个相对高级的话题(windows下也是如
此),在这个阶段讲要学习堆区域的bug.

\section{overflow using unlink}
\footnote{\url{https://sploitfun.wordpress.com/2015/02/26/heap-overflow-using-unlink/}}.

\section{overwrite using malloc}
\footnote{\url{https://sploitfun.wordpress.com/2015/03/04/heap-overflow-using-malloc-maleficarum/}}.

\section{off by one}
\footnote{\url{https://sploitfun.wordpress.com/2015/06/09/off-by-one-vulnerability-heap-based}}.

\section{UAF(use after free)}
\footnote{\url{https://sploitfun.wordpress.com/2015/06/16/use-after-free/}}.


% kernel安全
\chapter{内核}
 这个阶段归档了kernel安全相关的文档(安全保护,利用).
 \section{安全机制}
 早期kernel可以随意访问用户态代码,ret2usr技术可以让内核执行用户态的代码,不过随着
 Linux的发展SMAP(禁止kernel随意访问用户态,RFLAGE.AC标志位置位可以),SMEP禁止
 kernel态直接执行用户态代码.
 \subsection{SMAP}
现代Linux默认启用.
\subsection{SMEP/PXN}
现代Linux默认启用.
\subsection{kaslr}
ubuntu 14.04 desktop默认还没有启用,更多信息参考Ubuntu security Features
\footnote{\url{https://wiki.ubuntu.com/Security/Features\#Userspace_Hardening}}
\section{利用方法}
\subsection{rop-2-usr(废弃)}
早期能工作
\subsection{rop}

\subsection{vDSO overwriting}
SEMP using vDSO overwrites(CSAW Fianl 2015 string IPC)


% 漏洞挖掘
\chapter{漏洞挖掘}
漏洞挖掘的重要性不言而喻,打个比喻上面写的如何啃肉,漏洞挖掘就是肉在哪里.
\section{fuzz}

\subsection{why fuzz?}
- 容易实现
– 覆盖面广
– 低投入高产出

\subsection{why not fuzz?}
– 分析困难(无法调试)
– Panic多 / Exploitable少
– 欠缺精度

\subsection{where to fuzz?}
– ioctl
– sysctl
– File system
– Network
\subsection{how to fuzz?}

    

\section{代码审计}
\subsection{source}
 \begin{list}{\textbullet}{%
    \setlength\topsep{0pt} \setlength\partopsep{0pt}
    \setlength\parsep{0pt} \setlength\itemsep{0pt}
  }
    \item - Heap Overflow
    \item - Integer Overflow
    \item - Type Confusion
    \item - Use after Free
    \item - Logical Error
    \item - Kernel Information Leak
\end{list}

% 引用
\bibliography{reference}
\backmatter
\end{document}
% Local Variables:
% TeX-engine: xetex
% End:
