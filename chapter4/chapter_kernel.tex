\chapter{内核}
 这个阶段归档了kernel安全相关的文档(安全保护,利用).
 \section{安全机制}
 早期kernel可以随意访问用户态代码,ret2usr技术可以让内核执行用户态的代码,不过随着
 Linux的发展SMAP(禁止kernel随意访问用户态,RFLAGE.AC标志位置位可以),SMEP禁止
 kernel态直接执行用户态代码.
 \subsection{SMAP}
现代Linux默认启用.
\subsection{SMEP/PXN}
现代Linux默认启用.
\subsection{kaslr}
ubuntu 14.04 desktop默认还没有启用,更多信息参考Ubuntu security Features
\footnote{\url{https://wiki.ubuntu.com/Security/Features\#Userspace_Hardening}}
\section{利用方法}
\subsection{rop-2-usr(废弃)}
早期能工作
\subsection{rop}

\subsection{vDSO overwriting}
SEMP using vDSO overwrites(CSAW Fianl 2015 string IPC)
