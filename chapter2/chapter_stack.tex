\chapter{stack}
\par 这个阶段可能要花点时间了,需要学习主流的bypass安全机制的部分手段(base
stack).
\par ret2any: 返回到任何可以执行的地方, 已知的地方
\begin{list}{\textbullet}{%
    \setlength\topsep{0pt} \setlength\partopsep{0pt}
    \setlength\parsep{0pt} \setlength\itemsep{0pt}
  }
\item stack
\item data/heap
\item text
\item library (libc)
\item code chunk (ROP)
\end{list}

\section{CANARY}
\subsection{overwriting TLS}

\section{NX}
\subsection{return-to-libc}
\footnote{\url{https://sploitfun.wordpress.com/2015/05/08/bypassing-nx-bit-using-return-to-libc/}}.\newline
\subsection{chained return-to-libc} 
\footnote{\url{https://sploitfun.wordpress.com/2015/05/08/bypassing-nx-bit-using-chained-return-to-libc/}}.\newline

\section{ASLR}
\subsection{return-to-plt}
\footnote{\url{https://sploitfun.wordpress.com/2015/05/08/bypassing-aslr-part-i/}}.\newline
\subsection{brute-force}
\footnote{\url{https://sploitfun.wordpress.com/2015/05/08/bypassing-aslr-part-ii/}}.\newline
\subsection{overwriting GOT}
\footnote{\url{https://sploitfun.wordpress.com/2015/05/08/bypassing-aslr-part-iii/}}.\newline
