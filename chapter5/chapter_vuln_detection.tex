\chapter{漏洞挖掘}
漏洞挖掘的重要性不言而喻,打个比喻上面写的如何啃肉,漏洞挖掘就是肉在哪里.
\section{fuzz}

\subsection{why fuzz?}
- 容易实现
– 覆盖面广
– 低投入高产出

\subsection{why not fuzz?}
– 分析困难(无法调试)
– Panic多 / Exploitable少
– 欠缺精度

\subsection{where to fuzz?}
– ioctl
– sysctl
– File system
– Network
\subsection{how to fuzz?}

    

\section{代码审计}
\subsection{source}
 \begin{list}{\textbullet}{%
    \setlength\topsep{0pt} \setlength\partopsep{0pt}
    \setlength\parsep{0pt} \setlength\itemsep{0pt}
  }
    \item - Heap Overflow
    \item - Integer Overflow
    \item - Type Confusion
    \item - Use after Free
    \item - Logical Error
    \item - Kernel Information Leak
\end{list}